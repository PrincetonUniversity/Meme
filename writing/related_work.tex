\section{Related Work} \label{sec:related}
Software-Defined IXPs have been gaining attention recently~\cite{sdx-nsf}\cite{nildata}, with exchange point operators seeking testbeds and real deployments. Google's Cardigan project was deployed in a live internet exchange~\cite{stringer2014cardigan}, but Cardigan does not appear to use any compression techniques and thus we assume that the inflation factor of their forwarding tables is comparable to the ``naive" case we discuss, implying the number of forwarding entries would be far too large for available commodity switches when given reasonably-sized policies. The original SDX~\cite{sdx} paper introduced the concept of compression of routing information into the forwarding plane with the MDS algorithm for labeling packets which are subject to equivalent forwarding rules. It also introduced the idea of using the MAC address to store metadata, by advertising metadata as next-hop addresses. Despite these improvements, the project was unable to meet the scale or speed requirements of even moderately-sized IXPs. The recent iSDX~\cite{isdx} paper presents the full system which uses the compression technique detailed in this paper. In addition to a brief description of the technique, the work also discusses partitioning of outbound and inbound flow rules to avoid additional inflation, as well as tagging of packets to ensure participant flows are processed disjointly. 